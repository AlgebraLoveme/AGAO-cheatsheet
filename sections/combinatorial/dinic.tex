\section{Dinic's Algorithm for Maximum Flow}

Definition:
\begin{enumerate}
    \item \textbf{Level of a vertex}: given a source vertex $s$, the level of a vertex $u$ is defined to be the length of the shortest path from $s$ to $u$.
    \item \textbf{Admissible Edges}: an edge $(u, v)$ is called admissible if $l(u)+1=l(v)$, \ie, it is in one of the shortest paths from $s$ to $v$.
    \item \textbf{Level Graph}: the level graph of $G$ is the subgraph induced by only the admissible edges, \ie, only keep edges relevant to the shortest paths. Inferring the level graph is in $O(m)$.
    \item \textbf{Blocking flow}: a blocking flow in $G$ is a feasible flow in the level graph of $G$ such that (1) only uses admissible edges, and (2) saturates at least one edge for any $s$-$t$ path in the level graph of $G$, \ie, any $s$-$t$ path in the level graph is blocked by such a flow.
\end{enumerate}

\textbf{Dinic's algorithm}: starting from an empty flow, then repeatly augment the current flow by a blocking flow in the residual graph $G_f$ until no more $s$-$t$ path exists in $G_f$.

\subsection{\#Iterations of the Dinic's Algorithm}

At each iteration, the target vertex's level in the residual graph is increased by at least 1, as the original shortest path is blocked. As the level of any vertex is at most $n$, Dinic's algorithm terminates in $O(n)$ iterations. 

For unit-weight graphs, Dinic's algorithm can be proven to terminate in $O(\min\{m^{1/2}, n^{2/3}\})$ iterations. This is because now after $k$ iterations, the next iteration would erase at least $k$ edges, as the level of the target vertex is now at least $k$ in the residual graph. (1) This implies the value of the optimal flow cannot exceed $m/k$ as the optimal flow uses at least $k$ edges, which implies termination after at most another $m/k$ iterations. Setting $k=m^{1/2}$ gives the first bound. (2) By pigeonhole theorem, since the level graph has at most $n-1$ vertices, there are strictly more than $k/2$ of the level sets that has \#vertices less than $2n/k$. Using pigeonhole theorem again, there is at least two adjacent level set such that both have \#vertices less than $2n/k$. This implies there are at most $4n^2/k^2$ crossing edges between these two levels, and thus the algorithm terminates after at most another $4n^2/k^2$ iterations. Setting $k=2n^{2/3}$ gives the second bound.

\subsection{Finding Blocking Flow by Depth-First Seach}

Using depth-first search in the level graph, we are able to find a blocking flow in $O(nm)$, thus the total complexity of Dinic's algorithm is $O(n^2 m)$. In the unit-weight graph, depth-first seach is in $O(m)$, as each iteration either adds or removes one edge, thus the total complexity is $O(m \min\{m^{1/2}, n^{2/3}\})$.