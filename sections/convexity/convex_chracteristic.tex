\section{Convex Functions}

Definition:
\begin{enumerate}
    \item  A set $S \subseteq \mathbb{R}^{n}$ is called a convex set if any two points in $S$ contain their line, i.e. for any $\boldsymbol{x}, \boldsymbol{y} \in S$ we have that $\theta \boldsymbol{x}+(1-\theta) \boldsymbol{y} \in S$ for any $\theta \in[0,1]$.
    \item For a convex set $S \subseteq \mathbb{R}^{n}$, we say that a function $f: S \rightarrow \mathbb{R}$ is convex on $S$ if for any two points $\boldsymbol{x}, \boldsymbol{y} \in S$ and any $\theta \in[0,1]$ we have that
    $
    f(\theta \boldsymbol{x}+(1-\theta) \boldsymbol{y}) \leq \theta f(\boldsymbol{x})+(1-\theta) f(\boldsymbol{y}).
    $
\end{enumerate}

Properties:
\begin{enumerate}
    \item If $f$ is convex, then all its sub-levels $\{x: f(x)\le \alpha\}$ is convex. The reverse is not true, \eg, $x^3$.
\end{enumerate}

\subsection{First-order Characterization}
\textbf{Theorem}:
Let $S$ be an open convex subset of $\mathbb{R}^{n}$, and let $f: S \rightarrow \mathbb{R}$ be a differentiable function. Then, $f$ is convex if and only if for any $\boldsymbol{x}, \boldsymbol{y} \in S$ we have that $f(\boldsymbol{y}) \geq f(\boldsymbol{x})+$ $\boldsymbol{\nabla} f(\boldsymbol{x})^{\top}(\boldsymbol{y}-\boldsymbol{x}) .$

\subsection{Second-order Characterization}
\textbf{Theorem}:
Let $S \subseteq \mathbb{R}^{n}$ be open and convex, and let $f: S \rightarrow \mathbb{R}$ be twice continuously differentiable.
\begin{enumerate}
    \item $H_{f}(\boldsymbol{x})$ is positive semi-definite for any $\boldsymbol{x} \in S$ $\Leftrightarrow$ $f$ is convex on $S$.
    \item If $H_{f}(\boldsymbol{x})$ is positive definite for any $\boldsymbol{x} \in S$ then $f$ is strictly convex on $S$. The opposite is not true, \eg, for $f(x)=x^4$ at $x=0$.
\end{enumerate}

