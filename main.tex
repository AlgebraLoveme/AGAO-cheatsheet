\documentclass[11pt,landscape,a4paper]{article}
\usepackage[utf8]{inputenc}
\usepackage[T1]{fontenc}
%\usepackage[LY1,T1]{fontenc}
%\usepackage{frutigernext}
%\usepackage[lf,minionint]{MinionPro}
\usepackage{tikz}
\usetikzlibrary{shapes,positioning,arrows,fit,calc,graphs,graphs.standard}
\usepackage[nosf]{kpfonts}
\usepackage[t1]{sourcesanspro}
\usepackage{multicol}
\usepackage{wrapfig}
\usepackage[top=4mm,bottom=4mm,left=4mm,right=4mm]{geometry}
\usepackage[framemethod=tikz]{mdframed}
\usepackage{microtype}
\usepackage{pdfpages}
\usepackage[shortlabels]{enumitem}
\usepackage{flushend}
\usepackage[abbreviations]{foreign}
\usepackage{tcolorbox}
\usepackage{amsthm}
\usepackage{caption}


\raggedend 

\newif\iflong
\longtrue % \longfalse for short version and \longtrue for long version
\newcommand{\inLongVersion}[1]{\iflong #1\fi}


\setlist{nosep}
\setlist[itemize]{leftmargin=*}
\setlist[enumerate]{leftmargin=*}

\newcommand{\score}{\text{score}}
\newcommand{\encoder}{\text{encoder}}
\newcommand{\decoder}{\text{decoder}}
\newcommand{\E}{\mathbb{E}}
\newcommand{\Dist}{\mathcal{D}}
\newcommand{\normal}{\mathcal{N}}
\newcommand{\prob}{\mathbb{P}}
\newcommand{\HEADER}[1]{\begin{tcolorbox}
    \centering
    #1
\end{tcolorbox}}

\DeclareMathOperator{\cov}{\textbf{Cov}}
\DeclareMathOperator{\var}{\textbf{Var}}
\DeclareMathOperator{\argmin}{\textbf{argmin}}
\DeclareMathOperator{\argmax}{\textbf{argmax}}
\DeclareMathOperator{\sgn}{\textbf{sgn}}
\DeclareMathOperator{\dir}{\textbf{Dir}}
\DeclareMathOperator{\cat}{\textbf{Cat}}
\DeclareMathOperator{\vol}{\textbf{vol}}
\DeclareMathOperator{\nnz}{\textbf{nnz}}
\DeclareMathOperator{\Tr}{\textbf{Tr}}
\DeclareMathOperator{\im}{\textbf{im}}
\DeclareMathOperator{\val}{\textbf{val}}






\let\bar\overline

\include{./def}

\begin{document}
%\footnotesize
\small
\begin{multicols*}{3}

% Part 1: Convex Optimization

\HEADER{Convex Optimization}

% Part 2: spectral graph theory
\section{A Characterization of Convex Functions}
\textbf{Theorem:}
Let $S$ be an open convex subset of $\mathbb{R}^{n}$, and let $f: S \rightarrow \mathbb{R}$ be a differentiable function. Then, $f$ is convex if and only if for any $\boldsymbol{x}, \boldsymbol{y} \in S$ we have that $f(\boldsymbol{y}) \geq f(\boldsymbol{x})+$ $\boldsymbol{\nabla} f(\boldsymbol{x})^{\top}(\boldsymbol{y}-\boldsymbol{x}) .$
\section{Convexity and Second Derivatives,
Gradient Descent and Acceleration}
\textbf{Theorem:} (The Spectral Theorem for Symmetric Matrices). For all symmetric $\boldsymbol{A} \in$ $\mathbb{R}^{n \times n}$ there exist $\boldsymbol{V} \in \mathbb{R}^{n \times n}$ and a diagonal matrix $\boldsymbol{\Lambda} \in \mathbb{R}^{n \times n}$ s.t.
1. $\boldsymbol{A}=\boldsymbol{V} \boldsymbol{\Lambda} \boldsymbol{V}^{\top}$.
2. $\boldsymbol{V}^{\top} \boldsymbol{V}=\boldsymbol{I}$ (the $n \times n$ identity matrix). I.e. the columns of $\boldsymbol{V}$ form an orthonormal basis. Furthermore, $\boldsymbol{v}_{i}$ is an eigenvector of $\lambda_{i}(\boldsymbol{A})$, the ith eigenvalue of $\boldsymbol{A}$.
3. $\boldsymbol{\Lambda}_{i i}=\lambda_{i}(\boldsymbol{A})$.

\textbf{Theorem}:
Let $S \subseteq \mathbb{R}^{n}$ be open and convex, and let $f: S \rightarrow \mathbb{R}$ be twice continuously differentiable.
1. If $H_{f}(\boldsymbol{x})$ is positive semi-definite for any $\boldsymbol{x} \in S$ then $f$ is convex on $S$.
2. If $H_{f}(\boldsymbol{x})$ is positive definite for any $\boldsymbol{x} \in S$ then $f$ is strictly convex on $S$.
3. If $f$ is convex, then $H_{f}(\boldsymbol{x})$ is positive semi-definite $\forall \boldsymbol{x} \in S$.

\section{Gradient Descent}
\textbf{Theorem} When running Gradient Descent as given by the step in Equation (3.1), for all $i\left\|\boldsymbol{x}_{i}-\boldsymbol{x}^{*}\right\|_{2} \leq\left\|\boldsymbol{x}_{0}-\boldsymbol{x}^{*}\right\|_{2}$.


\textbf{Theorem:}Let $f: \mathbb{R}^{n} \rightarrow \mathbb{R}$ be a $\beta$-gradient Lipschitz, convex function. Let $\boldsymbol{x}_{0}$ be a given starting point, and let $\boldsymbol{x}^{*} \in \arg \min _{x \in \mathbb{R}^{n}} f(\boldsymbol{x})$ be a minimizer of $f$. The Gradient Descent algorithm given by
$$
\boldsymbol{x}_{i+1}=\boldsymbol{x}_{i}-\frac{1}{\beta} \boldsymbol{\nabla} f\left(\boldsymbol{x}_{i}\right)
$$
ensures that the kth iterate satisfies
$$
f\left(\boldsymbol{x}_{k}\right)-f\left(\boldsymbol{x}^{*}\right) \leq \frac{2 \beta\left\|\boldsymbol{x}_{0}-\boldsymbol{x}^{*}\right\|_{2}^{2}}{k+1} .
$$
\
\textbf{Theorem} Let $f: \mathbb{R}^{n} \rightarrow \mathbb{R}$ be a $\beta$-gradient Lipschitz, convex function. Let $\boldsymbol{x}_{0}$ be a
The Accelerated Gradient Descent algorithm given by
$$
\begin{aligned}
a_{i} &=\frac{i+1}{2}, A_{i}=\frac{(i+1)(i+2)}{4} \\
\boldsymbol{v}_{0} &=\boldsymbol{x}_{0}-\frac{1}{2 \beta} \boldsymbol{\nabla} f\left(\boldsymbol{x}_{0}\right) \\
\boldsymbol{y}_{i} &=\boldsymbol{x}_{i}-\frac{1}{\beta} \boldsymbol{\nabla} f\left(\boldsymbol{x}_{i}\right) \\
\boldsymbol{x}_{i+1} &=\frac{A_{i} \boldsymbol{y}_{i}+a_{i+1} \boldsymbol{v}_{i}}{A_{i+1}} \\
\boldsymbol{v}_{i+1} &=\boldsymbol{v}_{i}-\frac{a_{i+1}}{\beta} \boldsymbol{\nabla} f\left(\boldsymbol{x}_{i+1}\right)
\end{aligned}
$$
ensures that the kth iterate satisfies
$$
f\left(\boldsymbol{x}_{k}\right)-f\left(\boldsymbol{x}^{*}\right) \leq \frac{2 \beta\left\|\boldsymbol{x}_{0}-\boldsymbol{x}^{*}\right\|_{2}^{2}}{(k+1)(k+2)} .
$$


\HEADER{Spectral Graph Theory}

\emph{This part is organized as follows: the first two sections discuss graph Laplacian in the spectral domain, by bounding its eigenvalues and introducing its importance (Cheeger's inequality). The random walk section provides a use case of graph spectral in the analysis of convergence, and introduces the importance of Laplacian linear equations. The next two sections show how to solve it exactly by applying pseudo inverse, as the Laplacian is non-invertible. The rest sections show how to approximate a graph Laplacian and solve the Laplacian linear equation approximately and more efficiently.}

\input{sections/spectral/eigen_bound}

\section{Conductance}

Definitions:
\begin{enumerate}
    \item \textbf{Conductance of a vertex subset}: Given $\emptyset \subset S \subset V$, the conductance $\phi(S) := \phi(S)=\frac{|E(S, V \backslash S)|}{\min \{\vol(S), \vol(V \backslash S)\}}$, where $\vol(S) := \sum_{v \in S} \text{degree}(v)$. Define $\boldsymbol{1}_S$ to be the $n$-dimensional vector with only 1 for the vertices of $S$ and 0 for the vertices of $V \setminus S$. Assuming $\vol(S) \le \vol(V)/2$, thus $|E(V, V\setminus S)|= \sum_{(u,v)\in E} (\boldsymbol{1}_S(u) - \boldsymbol{1}_S(v))^2 = \boldsymbol{1}_S^\top L \boldsymbol{1}_S$ and $\vol(S) = \boldsymbol{1}_S^\top D \boldsymbol{1}_S$. Then $\phi(S) = \frac{\boldsymbol{1}_S^\top L \boldsymbol{1}_S}{\boldsymbol{1}_S^\top D \boldsymbol{1}_S}$.

    \item \textbf{Conductance of a graph}: The conductance $\phi(G) := \min_{\emptyset \subset S \subset V} \phi(S) = \min_{\substack{\emptyset \subset S \subset V \\ \vol(S)\le\vol(V)/2}} \phi(S).$

    \item \textbf{$\phi$-expander}: For any $\phi \in (0,1]$, we call a graph $G$ to be a $\phi$-expander if $\phi(G)\ge \phi$.
    \item \textbf{$\phi$-expander decomposition of quality $q$}: A partition $\{X_i\}$ of the vertex set $V$ is called a $\phi$-expander decomposition of quality $q$ if (1) each induced graph $G[X_i]$ is a $\phi$-expander, and (2.i) \#edges not contained in any $G[X_i]$ is at most $q\cdot\phi\cdot m$. The second condition is equivalent to (2.ii) The partition removes at most $q\cdot\phi\cdot m$ edges.
    \item \textbf{Normalized Laplacian}: We define the \textit{normalized Laplacian} to be $N := D^{-1/2} L D^{-1/2}$. $N$ is still PSD, with first eigenvalue equals 0 associated with eigenvector $D^{1/2}\boldsymbol{1}$. By Courant-Fischer theorem, $\lambda_2(N) = \min_{x \perp D^{1/2}\boldsymbol{1}} \frac{x^\top N x}{x^\top x} = \min_{z \perp d} \frac{z^\top L z}{z^\top D z}$.
\end{enumerate}

\subsection{Cheeger's Inequality}

Notice that the $\lambda_2(N)$ has similar forms to $\phi(G)$. Cheeger's inequality aims to bound $\phi(G)$ by $\lambda_2(N)$.

\textbf{Cheeger's Inequality}: $\frac{\lambda_2(N)}{2} \le \phi(G) \le \sqrt{2\lambda_2(N)}$.

The lower bound is proved by restricting the minimum in $\lambda_2(N)$ to be $z_S = \boldsymbol{1}_S - \alpha \boldsymbol{1}$ for some $\alpha$ such that $z_S \perp d$. The upper bound is proved by constructing $S$ for any $z \perp d$ such that $\frac{\boldsymbol{1}_S^\top L \boldsymbol{1}_S}{\boldsymbol{1}_S^\top D \boldsymbol{1}_S} \le \sqrt{2 \frac{z^\top L z}{z^\top D z}}$.



\section{Random Walks on a Graph}

A \textbf{random walk on a graph $G$} is a Markov Chain with transition probability $\prob(v_{t+1}=v \mid v_t=u)= w(u,v)/d(u)$ iff $(u,v)\in E$ and 0 otherwise. The transition matrix is thus $W=AD^{-1} = I - D^{1/2} N D^{-1/2}$ and $p_t = W^t p_0$. Define $\boldsymbol{\pi}=\frac{d}{\boldsymbol{1}^\top d}$, thus $\boldsymbol{\pi} = W \boldsymbol{\pi}$ for any $G$, so every $G$ has a stationary distribution.

\subsection{Lazy Random Walks}

A \textbf{lazy random walk on a graph $G$} is a random walk, but has half probability to not move for every step. Assuming that $G$ is connected, the lazy random walk guarantees ergodicity of the Markov Chain, and thus convergence to the stationary distribution. The transition matrix is $\tilde{W} = \frac{1}{2}(I+W) = I - \frac{1}{2} D^{1/2} N D^{-1/2}$.

\textbf{Relation between lazy random walk and normalized Laplacian}: For the $i$-th eigenvalue $\nu_i$ of $N$ associated with eigenvector $\psi_i$, the $\tilde{W}$ has an eigenvalue $1-\frac{1}{2}\nu_i$ associated with eigenvector $D^{1/2}\psi_i$. Since $0 \preceq L \preceq 2D$, we have $0 \preceq N \preceq 2\boldsymbol{I}$ and thus $0\le \lambda_i(N) \le 2$. Therefore, we conclude that all eigenvalues of $\tilde{W} \in [0,1]$.

\textbf{Dynamics of lazy random walk}: Expanding the starting distribution $p_0$ by the eigenvectors of $\tilde{W}$, we have for some $\{\alpha_i\}$ that $p_0 = \sum_{i=1}^n \alpha_i D^{1/2} \psi_i$. Therefore, we have $p_t = \tilde{W}^t p_0 = \sum_{i=1}^n \alpha_i (1-\frac{1}{2}\nu_i)^t D^{1/2}\psi_i \rightarrow \alpha_1 D^{1/2}\psi_1$ as $\nu_1=0$ and $\nu_i>0$ for $i\ne 1$. Since $\psi_1 \propto D^{1/2}\boldsymbol{1}$, we have $\psi_1 = \frac{d^{1/2}}{(\boldsymbol{1}^\top d)^{1/2}}$, thus $\alpha_1 = \psi_1^\top D^{-1/2} p_0 = \frac{\boldsymbol{1}^\top p_0}{(\boldsymbol{1}^\top d)^{1/2}} = \frac{1}{(\boldsymbol{1}^\top d)^{1/2}}$ and $\alpha_1 D^{1/2} \psi_1 = \pi$, which implies $p_t \rightarrow \pi$, the stationary distribution.

\textbf{Rate of Convergence}: For any unit-weight connected graph $G$ and any starting distribution $p_0$, we have $\|p_t - \pi\|_\infty \le e^{-\nu_2 t/2} \sqrt{n}$. Therefore, a larger $\nu_2$ and smaller vertex set means faster convergence, and the convergence rate is exponential. This can be viewed as larger $\nu_2$ implies larger conductance by Cheeger's inequality, which means better connectedness.

For complete graph $K_n$ which has $\lambda_2(N) = \Theta(1)$ as implied by Cheeger's inequality, the steps required for $\epsilon$-convergence is $O(\log(n/\epsilon))$. For path graph which has $\lambda_2(N) =\Omega(1/n^2)$ as implied by Cheeger's inequality, the steps required for $\epsilon$-convergence is $O(n^2 \log (n/\epsilon))$.
\subsection{Hitting Time}

\textbf{The expected hitting time} from $a$ to $s$ is defined by $\E H_{a,s}$, where $H_{a,s} = \argmin_t\{v_t=s\mid v_0=a\}$. We want $\E H_{a,s}$ for all vertices $a$ and denote the vector as $h$, \eg, $h(s)=0$.

By one-step analysis, we have $h(a) = 1+\sum_{(a,b)\in E} \frac{w(a,b)}{d(a)} h(b) = 1+\boldsymbol{1}_a^\top W^\top h$, and thus $1 = \boldsymbol{1}_a^\top (\boldsymbol{I}-W^\top)h$. Combining the equation for all vertices except $s$, we have $\boldsymbol{1} - \alpha \boldsymbol{1}_s = (\boldsymbol{I}-W^\top)h$, where $\alpha$ represents the extra freedom from the $n-1$ equations. Multiplying both side by $D$, we get $d - \alpha d(s) \boldsymbol{1}_s = (D-A)h = Lh$, which only have solution when $d - \alpha d(s) \boldsymbol{1}_s \perp \boldsymbol{1}$. Therefore, $\alpha = \|d\|_1 / d(s)$. 

To summarize, by solving $Lh = d - \|d\|_1 \boldsymbol{1}$, we can get the expected hitting time from all vertices to $s$. Note that the solution has one extra freedom because $\dim(\ker(L))=1$, and the correct expected hitting time is $h - h(s)\boldsymbol{1}$ to enforce the constraint that $h(s)=1$. The equation can be solved in $\tilde{O}(m)$.

\section{Pseudo-Inverse and Effective Resistance}

Given a Laplacian $L$, its (Moore-Penrose) pseudo inverse is defined to be either of the two equivalents:
\begin{enumerate}
    \item A matrix $L^+$ that is (1) symmetric, (2) $L^+ v=0$ for $v \in \ker(L)$, and (3) $L^+ L v = L L^+ v = v$ for  $v \in \ker(L)$.
    \item Let $\lambda_i, v_i$ be the $i$-th eigenvalue and eigenvector. Then $L^+ = \sum_{\lambda_i \ne 0} \lambda_i^{-1} v_i v_i^\top$.
\end{enumerate}

Property:
\begin{itemize}
    \item Assume $M=X Y X^\top$, where $X$ is real and invertible, and $Y$ is real and symmetric. Let $\Pi_M$ be the orthogonal projection to the image of $M$. Then $M^+ = \Pi_M (X^\top)^{-1} Y^+ X^{-1} \Pi_M$.
    \item For symmetric $L$, $\Pi_L:= \sum_{\lambda_i \ne 0} v_i v_i^\top =  L^{+/2} L L^{+/2} = L^+ L = L L^+$ is the orthogonal projection to the image of $L$, \ie, $\Pi_L v = 0$ for any $v \in \ker(L)$ and $\Pi_v = v$ for any $v \in \im(L)$. For connected $G$, $\Pi_{L} = \boldsymbol{I} - \frac{1}{n} \boldsymbol{1} \boldsymbol{1}^\top$.
\end{itemize}

The effective resistance between vertex $a$ and $b$ is defined to be the cost (energy lost) to routing one unit (of positive electric charge) from $a$ to $b$: $R_{\text{eff}}(a,b) = \min_{Bf = \boldsymbol{1}_b - \boldsymbol{1}_a} f^\top R f = \tilde{f}^\top R \tilde{f}$, where $\tilde{f}$ is the electric flow. Let $\tilde{x}$ be the electric voltages, we also have $L\tilde{x} = \boldsymbol{1}_b - \boldsymbol{1}_a$, and thus $R_{\text{eff}}(a,b) = \tilde{x}^\top L \tilde{x} = (\boldsymbol{1}_b - \boldsymbol{1}_a)^\top L^+ (\boldsymbol{1}_b - \boldsymbol{1}_a) = \|L^{+/2} (\boldsymbol{1}_b - \boldsymbol{1}_a)\|_2^2$.

Effective Resistance is a distance defined on the vertex pairs, i.e. $R_{\text{eff}}(a,c) \le R_{\text{eff}}(a,b) + R_{\text{eff}}(b,c)$.

\section{Gaussian Elimination for Laplacian}



\section{Concentration of Random Matrices}

\section{Solving Laplacian Linear Equations Approximately}

Idea: solving Laplacian linear equations requires $O(n^3)$ to get the Cholesky decomposition, which is expensive when the graph is large. By approximating the Laplacian, we can get an approximation of the solution quickly, especially in sparse graphs.

Given PSD matrix $M$ and $d \in \im(M)$, let $M x^* = d$. We say that $\tilde{x}$ is an $\epsilon$-approximate solution to $Mx=d$ iff $\|\tilde{x} - x^*\|_M^2 \le \epsilon \|x^*\|_M^2$, where $\|x\|_M^2 = x^\top M x$. Note that any solution to $M x = d$ has the same $\|\cdot\|_M^2$, as they differ by a vector in the kernel of $M$.

\textbf{Theorem}: Given a Laplacian $\boldsymbol{L}$ of a weighted undirected graph $G=(V, E, \boldsymbol{w})$ with $|E|=m$ and $|V|=n$ and a demand vector $\boldsymbol{d} \in \mathbb{R}^{V}$, we can find $\tilde{\boldsymbol{x}}$ that is an $\epsilon$-approximate solution to $\boldsymbol{L x}=\boldsymbol{d}$, using an algorithm that takes time $O\left(m \log ^{c} n \log (1 / \epsilon)\right)$ for some fixed constant $c$ and succeeds with probability $1-1 / n^{10}$. Note that without knowing the Cholesky decomposition in advance, the exact solution requires $O(n^3)$ and $m \le n^2 / 2$.

Idea: during the exact Cholesky decomposition, a clique is added to the graph every time. With sampling, we can get a sparse approximation of such cliques and add these approximated cliques instead.

% Part 3: Combinatorial Graph Algorithms

\HEADER{Combinatorial Graph Algorithms}

\section{Maximum Flows and Minimum Cuts}

\subsection{Flows}
Definition:
\begin{enumerate}
    \item \textbf{$s$-$t$ flow}: A $s$-$t$ flow is a flow such that $\boldsymbol{B}\boldsymbol{f} = F (-\boldsymbol{1}_s + \boldsymbol{1}_t)$ for some $F\ge 0$, \ie, routes some unit from $s$ to $t$.
    \item \textbf{Maximum flow problem}: Given a source vertex and a sink vertex, maximize $F$ such that $\boldsymbol{B}\boldsymbol{f} = F (-\boldsymbol{1}_s + \boldsymbol{1}_t)$ and $0 \le \boldsymbol{f} \le \boldsymbol{c}$.
    \item \textbf{Path flow}: A $s$-$t$ path flow is a $s$-$t$ flow that only uses a simple path from $s$ to $t$.
    \item \textbf{Cycle flow}: A cycle flow is a flow that only uses a simple cycle, it does not create net-in or net-out.
\end{enumerate}

\textbf{Path-cycle decomposition lemma}: Any $s$-$t$ flow can be decomposed to a sum of $s$-$t$ path flows and cycle flows such that the summation has at most $\nnz(\boldsymbol{f})$ terms.

There is always an optimal flow that can be decomposed to only path flows, as the cycle flow does not route anything from $s$-$t$ and removing all cycle flows in an optimal flow creates another optimal flow.

\subsection{Cuts}

Definition:
\begin{enumerate}
    \item \textbf{$s$-$t$ cuts}: A $s$-$t$ cut is a cut $(S, V\setminus S)$ such that $s \in S$ and $t \in V\setminus S$.
    \item \textbf{Minimum cut problem}: Given two vertices $s$ and $t$, minimize the cut value $c_G(S)=\sum_{e \in E \cap (S\times V\setminus S)} w(e)$ such that $s \in S$ and $t \in V\setminus S$.
\end{enumerate}

If there is no feasible $s$-$t$ flow, then define $S$ to be the set of vertices reachable from $s$, $(S, V\setminus S)$ is a $s$-$t$ cut.

\subsection{Duality of Max Flow and Min Cut}

\section{Dinic's Algorithm for Maximum Flow}

\section{Link-Cut Trees}

Definition:
\begin{enumerate}
    \item \textbf{Dynamic Graph}: a graph that is constantly changing by edge insertion/deletion. No vertex changes.
    \item \textbf{Dynamic rooted forest}: for every edge change, the graph remains a directed forest, and each tree in the forest has a single root. The root can be reached from any vertex in this tree.
\end{enumerate}

A link-cut tree is a data structure that speeds up dynamic rooted forest changes, \ie, it always represents a uniquely determined dynamic rooted forest, but can execute edge changes in less amortized time. This can be used to speed up the process of finding blocking flows, thus making the Dinic's algorithm faster. \textbf{Note}: the link-cut tree is designed to carry weight on its vertices but not edges. However, any edge-weighted graph can be converted to be vertex-weighted, by adding a dummy vertex in the middle of each edge with the same weight, and setting the weight of all original vertices to be $+\infty$. We choose $+\infty$ so that this does not change the max flow. Other values may be chosen for other usages.

The link-cut tree supports the following operations:
\begin{enumerate}
    \item \emph{Initialize($G$)}: creates a link-cut tree that refers to an empty dynamic rooted forest with the same vertices of $G$ but no edges, \ie, every vertex is its own root.
    \item \emph{FindRoot($v$)}: find the root of $v$ in the current dynamic rooted forest.
    \item \emph{AddCost($v$, $\Delta$)}: add $\Delta$ to the cost of every vertex on the path from $v$ to its root.
    \item \emph{FindMin($v$)}: returns the first min-cost vertex on the path from $v$ to its root and its associated cost.
    \item \emph{Link($u$, $v$)}: add an edge $(u, v)$, assuming $u$ to be a root vertex and $v$ to be in another tree. Note that the required property maintains the graph to be rooted forest and merges two trees into one.
    \item \emph{Cut($u$, $v$)}: cuts a current edge $(u,v)$. This splits one tree into two, with $u$ being one of the root.
\end{enumerate}

\textbf{Theorem}: The link-cut tree can realize any sequence of $m$ operations in total expected time $O(m\log^2 n + n)$.

\subsection{Implementation of Link-Cut Trees}

The implementation relies on the treap structure (basically search property of binary search trees for one key + heap order for another independent key). Basically, we first construct these operations restricted to path trees, encoded by balanced treaps. The path is encoded such that one key (the searching key) of the treap stores the ``order'', \ie, $v$ is always at the right of $u$ if $u$ is the ancestor of $v$, and the other key (the heap key) stores a random value for constructing balanced binary trees with high probability.

The weight changes are boosted by associating the difference between the min-cost of current vertex and the min-cost of its parent, and the difference between the cost of current vertex and the min-cost of it. We first call PCut and Plink, if necessary, to make $v$ have no precessor. When the current vertex has no precessor, the PathAddCost only needs to adjust the root's min-cost, and the PFindMin only needs to follow the child with $\Delta \text{min}=0$. As the depth is $O(\log n)$, these operations are $O(\log n)$ as well.

To implement the general link-cut tree, we decompose each tree into paths so that each vertex only occur in exactly one path and each internal vertex has exactly one incoming edge. By swtiching between the different path decompositions (requires $O(\log n)$), we are able to make sure the tree under changing is always a path. This is possible because all these operations actually only changes a specific path.

\subsection{Boosting Blocking Flows by Link-Cut Trees}

First, as described before, we convert the level graph of current residual graph $G_f$ into vertex-weighted by adding dummy vertices. The change is that we now use the operations provided by the link-cut tree to do the DFS, which is faster than the naive DFS. The modified DFS uses Link and Cut to maintain the search path, and use AddCost to extract a feasible flow found. As there are $O(m)$ operations in the DFS, we can find the blocking flow in $O(m\log^2 n + n)$.

\section{The Cut-Matching Game}
Definition:
\begin{enumerate}
    \item \textbf{Sparsity of a vertex subset}: Given $\emptyset \subset S \subset V$, the conductance $\psi(S) := \frac{|E(S, V \backslash S)|}{\min \{|S|, |V\setminus S|\}}$. This is different to the conductance $\phi(S)$ in the denominator. Since $\vol{S} \ge |S|$, it is guaranteed that $\psi(S) \ge \phi(S)$.
    \item \textbf{Sparsity of a graph}: $\psi(G):= \min_{\emptyset \subset S \subset V} \psi(S)$. We say $G$ is a $\psi$-expander w.r.t. sparsity iff $\psi(G) \ge \psi$. The cut that achieves the minimum is called the sparsest cut.
\end{enumerate}

The cut-matching game is an algorithm that involves interaction of the cut player and the matching player, designed to follow a specific strategy, so that the result of such a game could certfify the sparsity of a graph.

\subsection{Certifying via Embedding}

Given graphs $H$ and $G$ defined on the same vertex set, we say a function is an embedding of $H$ into $G$ if it maps each edge $(u,v) \in H$ to a $u$-to-$v$ path in $G$. We define the congestion of such an embedding to be the maximum number of times that any edge in $G$ appears on any embedding path.

\textbf{Property}: given a $\psi$-expander graph $H$ and an embedding of $H$ into $G$ with congestion $C$, then $G$ is a $\psi/C$-expander.
\begin{proof}
    For any cut $(S, V\setminus S)$ such that $|S|\le n/2$, we have $|E_H(S, V\setminus S)| \ge \psi |S|$. For every edge $(u,v)$ in $|E_H(S, V\setminus S)|$, there is a path from $u$ to $v$ in $G$ which crosses the cut. Since each edge crossing the cut can be used at most $C$ times, we have that $|E_G(S, V\setminus S)| \ge \psi |S|/C$, which implies that $G$ is a $\psi/C$-expander.
\end{proof}

\subsection{Certifying $\psi$-expanders via Max Flows}

Theorem 14.1.1. There is an algorithm CertifyOrCut$(G, \psi)$ that given a graph $G$ and a parameter $0<\psi \leq 1$, either:
\begin{itemize}
    \item Certifies that $G$ is a $\Omega\left(\psi / \log ^{2} n\right)$-expander w.r.t. sparsity.
    \item Presents a cut $S$ such that $\psi(S) \leq O(\psi)$.
\end{itemize}
The algorithm runs in time $O\left(\log^{2} n\right) \cdot T_{\text {max\_flow }}(G)+\tilde{O}(m)$ where $T_{\text{max-flow }}(G)$ is the time it takes to solve a Max Flow problem on $G$.

Illustration of one iteration of the Algorithm:

\begin{center}
\includegraphics[width=.5\columnwidth]{imgs/cut-match.png}
\end{center}

 In a), a bi-partition $(S_i, \bar{S_i})$ of $V$ is found (requires random walk on $G$). In b), the bi-partition is used to obtain a flow problem where we inject one unit of flow to each vertex in $S_i$ via super-source $s$ and extract one unit of flow from each vertex in $\bar{S_i}$ via super-sink $t$. Every edge is set to have capacity $1/\psi$. Then we solve this problem to get a flow $f$ with $\val(f)=n/2$. If such flow does not exist, then we return the min-cut of this flow problem, removing the dummy source and sink.  c) If such flow exists, we construct a path flow decomposition. For each path, the first vertex is in $S$ and the last vertex in $\bar{S}$. d) We find $M_{i}$ to be the one-to-one matching between endpoints in $S$ and $\bar{S}$ defined by the path flows.

 It can be proven that after $T=\Theta(\log^2 n)$ iterations, the union of the $T$ matchings is a $1/2$-expander and $G$ can be embedded into $H$ with congestion $O(\log^2 n / \psi)$, which certifies that $G$ is a $O(\psi / \log^2 n)$-expander. If one of the iterations presents a cut, then it can be proven that the sparsity of this cut is $O(\psi)$.

\end{multicols*}
\end{document}
